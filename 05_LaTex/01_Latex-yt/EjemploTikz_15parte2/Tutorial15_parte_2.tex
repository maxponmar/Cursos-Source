\documentclass[12pt]{article}
\usepackage[lmargin=0.5cm,rmargin=1cm,
top=0.5cm,bottom=2cm]{geometry}
\usepackage[utf8]{inputenc}
\usepackage[T1]{fontenc}
\usepackage[spanish]{babel}
\parindent=0cm
%%
\usepackage{tikz}
%%

\begin{document}

\section{Tikz}
\subsection{Figuras junto a texto}
Iniciamos con una sencilla figura \tikz{\draw (0,0) rectangle (1,0.5);}.\\[0.5cm]

Ejemplo en tikz colocando
\tikz[scale=.25]{
\draw[help lines] (0,-1) grid (4,2);
\draw[fill=blue] (0,0) rectangle (2,1);
\fill[ball color=red] (3,0) circle (1);
}
 figuras junto a texto.
\tikz{
\draw (0,0) -- (2,0);
}
\vspace{0.5cm}

\tikz[scale=.25]{
\draw[help lines] (0,-1) grid (4,2);
\draw[fill=blue] (0,0) rectangle (2,1);
\fill[ball color=red] (3,0) circle (1);
}
Al inicio del texto también se pueden colocar figuras
\tikz[scale=.25]{
\fill[ball color=red] (3,0) circle (1);
}
al centro, y también al final.
\tikz[scale=.25]{
\draw[fill=blue] (0,0) rectangle (2,1);
}

\subsection{Lineas}
\subsubsection{Lineas continuas}

\begin{tikzpicture}
\draw[line width=5pt](0,3) -- (3,3);

\draw[ultra thick](0,2.5) -- (3,2.5);
\draw[line width=1.6pt](0,2) -- (3,2);

\draw[thin](0,1.5) -- (3,1.5);
\draw[line width=0.4pt](0,1) -- (3,1);

\draw[ultra thin](0,0.5) -- (3,0.5);
\draw[line width=0.1pt](0,0) -- (3,0);
\end{tikzpicture}

\subsubsection{Lineas punteadas}

\begin{tikzpicture}
\draw[dotted](0,1) -- (3,1);
\draw[densely dotted](0,0.5) -- (3,0.5);

\draw[dashed](0,0) -- (3,0);
\end{tikzpicture}































\end{document}