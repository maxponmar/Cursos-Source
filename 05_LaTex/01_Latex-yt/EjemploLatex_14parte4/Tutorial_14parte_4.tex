\documentclass{beamer}
\usepackage[T1]{fontenc}
\usepackage[utf8]{inputenc}
\usepackage[spanish,es-tabla]{babel}
\usepackage{amsmath}
\usepackage{amssymb,amsfonts,latexsym,cancel}
\usepackage{float}
\usepackage{graphicx}
\usepackage{epstopdf}
\usepackage{subfigure}
\usetheme{Warsaw}
%\usetheme{Berkeley}
\setbeamercovered{transparent}

\title[Beamer]{\bf\Huge Beamer básico}
\subtitle{Tutorial 14 parte 4}
\author[Héctor Misael Bacilio Navarro]{Héctor Misael Bacilio Navarro}
\institute[]{Universidad de Guadalajara}
\date[2017]{\scriptsize{2017}}
%\logo{\includegraphics[width=1.0cm]{figuras/udg}}
\titlegraphic{\includegraphics[width=1.0cm]{figuras/udg}}
\begin{document}

\begin{frame}
\titlepage
\end{frame}

\begin{frame}
\frametitle{Contenido de la presentación}
\tableofcontents
\end{frame}

\section{Listas}

\begin{frame}
\frametitle{Listas}
\framesubtitle{itemize}

\begin{itemize}
\item El primer elemento
\item El segundo elemento
\item y continuamos
\end{itemize}

\end{frame}


\begin{frame}
\frametitle{Listas}
\framesubtitle{enumerate}

\begin{enumerate}
\item El primer elemento
\item El segundo elemento
\item y continuamos
\end{enumerate}

\end{frame}

\begin{frame}{Overlays (velos) básicos}
\framesubtitle{Opción [<+->]}
Nos permite visualizar paso a paso cada elemento de la lista
\begin{enumerate}[<+->]
\item El primer elemento
\item El segundo elemento
\item y continuamos
\end{enumerate}
\end{frame}

\begin{frame}{Overlays (velos) básicos}
\framesubtitle{Pause}
Utilizando el comando pause
\begin{enumerate}
\item El primer elemento 
\item El segundo elemento \pause
\item y continuamos
\end{enumerate}
\end{frame}

\begin{frame}{Overlays (velos) básicos}
\framesubtitle{Opción <i->}
Tenemos un mejor control de que elemento de nuestra lista se visualiza
\begin{enumerate}
\item<1-> El primer elemento
\item<1-> El segundo elemento
\item<2-> y continuamos
\end{enumerate}
\end{frame}

\section{Alertas}
\begin{frame}{Overlays (velos) básicos}
\framesubtitle{Alertas <i-|alert@i>}
Nos muestra un efecto de alerta
\begin{enumerate}
\item<1-|alert@1> El primer elemento
\item<2-|alert@2> El segundo elemento
\item<3-|alert@3> El tercero elemento
\item<4-|alert@4> y continuamos
\end{enumerate}
\end{frame}

\begin{frame}{Overlays (velos) básicos}
\framesubtitle{Alertas <i-|alert@i> opción 2}
Nos muestra un efecto de alerta
\begin{enumerate}
\item<1-|alert@1> El primer elemento
\item<2-|alert@2> El segundo elemento
\item<1-|alert@1> El tercero elemento
\item<3-|alert@3> y continuamos
\end{enumerate}
\end{frame}

\begin{frame}{Overlays (velos) básicos}
\framesubtitle{Alertas [<+-|alert@+>] opción 2}
Nos muestra un efecto de alerta de forma general
\begin{enumerate}[<+-|alert@+>]
\item El primer elemento
\item El segundo elemento
\item El tercero elemento
\item y continuamos
\end{enumerate}
\end{frame}

\section{Bloques}
\begin{frame}{Bloques}
\framesubtitle{block}
Utilizamos bloques para destacar elementos

\begin{block}{Titulo}
Resaltamos la información
\end{block}
\pause
\begin{alertblock}{Destacar}
Resaltamos información
\end{alertblock}

\end{frame}


\end{document}